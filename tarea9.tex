\documentclass{article}

% formato
\usepackage[margin = 1.5cm, letterpaper]{geometry}
\usepackage[spanish,es-noshorthands]{babel}

%tablas
\usepackage{graphicx}

%formato ecuaciones
\usepackage{amsmath}

% símbolos
\usepackage{amssymb}

% manejo de tablas
\usepackage{float}

% autómatas
\usepackage{tikz}
\usetikzlibrary{automata, positioning, arrows}

\begin{document}
    \title{
        Autómatas y Lenguajes formales \\
        Ejercicio Semanal 9
    }

    \author{
        Sandra del Mar Soto Corderi \\
        Edgar Quiroz Castañeda
    }

    \date{
        11 de abril del 2019
    }
    
    \maketitle

    \begin{enumerate}
        \item {
        Responde a los incisos con base en el siguiente lenguaje $L = 
        \{a^nb^{2n} | n \in \mathbb{N}\}$
        \begin{enumerate}
        	\item {
        	Demuestra que $L$ no es regular.\\
            Demostraremos que el lenguaje no es regular usando el conjunto 
            estafador.
            Un conjunto infinito $S \subseteq \Sigma^*$ es un conjunto estafador
            para L si y sólo si $\forall x,y \in S (x \not\equiv_L y)$.\\
            Sea $S = \{ a^kb^k | k \in \mathbb{N}\}$, veamos que S es un 
            conjunto estafador: Sean $a^nb^n , a^mb^m \in S \ \text{con} \ n 
            \neq m$\\
        	Tomemos $x = b^n$\\
        	Por un lado tenemos $ a^nb^nb^n = a^nb^{2n} \in L$\\
        	Por otra parte tenemos $a^mb^mb^n = a^mb^{m + n} \not \in L$\\
            Por lo tanto $a^nb^n \not\equiv_L a^mb^m$ y S es un conjunto 
            estafador de L.\\
            Como pudimos encontrar un conjunto estafador de L, concluimos que 
            L no es regular. $\blacksquare$.
        	}
        	\item{
            Diseña un Autómata de Pila que acepte a $L$ con criterio de 
            aceptación de estado final.
            \[M = \langle \{q_{0}, q_{1}, q_{2}, q_{3}\}, \{a, b\}, \{A, B\},
            \delta, q_{0}, Z_{0}, \{q_{3}\}\rangle\]
            Con $\delta$ dada por
            \begin{align*}
                &\delta(q_{0}, \epsilon, Z_{0}) = (q_{3}, \epsilon) \\
                &\delta(q_{0}, a, \gamma) = (q_{0}, \gamma) \\
                &\delta(q_{0}, \epsilon, \gamma) = (q_{1}, \gamma) \\
                &\delta(q_{1}, b, A) = (q_{2}, A) \\
                &\delta(q_{1}, \epsilon, Z_{0}) = (q_{3}, \epsilon) \\
                &\delta(q_{2}, b, A) = (q_{1}, \epsilon)
            \end{align*}
            Con representación gráfica
        	\begin{figure}[H]
                \centering
                \begin{tikzpicture}[auto]
                    % estados del autómata
                    \node [state, initial] (q0) {$q_{0}$};
                    \node [state] (q1) [right=of q0] {$q_{1}$};
                    \node [state] (q2) [right=of q1] {$q_{2}$};
                    \node [state, accepting] (q3) [below=of q2] {$q_{3}$};
                    
                    % transiciones
                    \path[->]
                    (q0) edge [loop above] node {$a, \gamma / A\gamma$} (q0)
                         edge node {$b, A / A$} (q1)
                         edge node [below] {$\epsilon, Z_{0} / \epsilon$} (q3)
                    (q1) edge [bend right] node [below] {$b, A / \epsilon$} (q2)
                    (q2) edge node {$\epsilon, Z_{0} / \epsilon$} (q3)
                    (q2) edge [bend right] node [above] {$b, A / A$} (q1);
                \end{tikzpicture}
                \caption{Autómata que acepta a $L$ por estado final}
                \label{fib:pda1}
            \end{figure}
        	}
        	\item{
            Transforma el PDA del inciso anterior en uno con criterio de 
            aceptación por pila vacía.\\
            En $M$ la única manera de vaciar la pila es por medio de las
            transiciones que llevan a $q_{3}$, pues son las únicas que quitan a
            $Z_{0}$ de la pila.\\
            Entonces, si se vacía la pila, se llega a $q_{3}$.
            Pero $q_{3}$ es final, así que si una cadena vacía la pila, entonces
            llega a un estado final, y la cadena está en $L(M)$.\\
            Por otra parte, si se llegó a un estado final, la pila debe estar
            vacía, pues las úniacs transiciones que llevan al estado final la
            vacían. Así que si una cadena es aceptada por estado final, vacía la
            pila y está en $V(M)$. Por lo tanto, el lenguaje aceptado por estado
            final y por pila vacía es el mismo.\\
            Por lo que $M$ ya acepta por pila vacía, por lo que la tranformación
            de $M$ es sí mismo.
        	}
        \end{enumerate}
    	}
    \end{enumerate}
\end{document}