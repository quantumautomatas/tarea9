\documentclass{article}

% formato
\usepackage[margin = 1.5cm, letterpaper]{geometry}
\usepackage[utf8]{inputenc}

%tablas
\usepackage{graphicx}

%formato ecuaciones
\usepackage{amsmath}

% símbolos
\usepackage{amssymb}

% manejo de tablas
\usepackage{float}

\begin{document}
    \title{
        Autómatas y Lenguajes formales \\
        Ejercicio Semanal 9
    }

    \author{
        Sandra del Mar Soto Corderi \\
        Edgar Quiroz Castañeda
    }

    \date{
        11 de abril del 2019
    }
    
    \maketitle

    \begin{enumerate}
        \item {
        Responde a los incisos con base en el siguiente lenguaje $L = \{a^nb^{2n} | n \in \mathbb{N}\}$
        \begin{enumerate}
        	\item {
        	Demuestra que $L$ no es regular.\\
        	
        	Demostraremos que el lenguaje no es regular usando el conjunto estafador.
        	
        	Un conjunto infinito $S \subseteq \Sigma^*$ es un conjunto estafador para L si y sólo si $\forall x,y \in S (x \not\equiv_L y)$.\\
        	
        	Sea $S = \{ a^kb^k | k \in \mathbb{N}\}$, veamos que S es un conjunto estafador: Sean $a^nb^n , a^mb^m \in S \ \text{con} \ n \neq m$\\
        	
        	Tomemos $x = b^n$\\
        	Por un lado tenemos $ a^nb^nb^n = a^nb^{2n} \in L$\\
        	Por otra parte tenemos $a^mb^mb^n = a^mb^{m + n} \not \in L$\\
        	Por lo tanto $a^nb^n \not\equiv_L a^mb^m$ y S es un conjunto estafador de L.\\
        	
        	Como pudimos encontrar un conjunto estafador de L, concluimos que L no es regular. $\blacksquare$\\
        	
        	}
        	\item{
        	Diseña un Autómata de Pila que acepte a $L$ con criterio de aceptación de estado final.\\
        	
        	
        	}
        	\item{
        	Transforma el PDA del inciso anterior en uno con criterio de aceptación por pila vacía.
        	
        	}
        \end{enumerate}
    	}
    \end{enumerate}
\end{document}